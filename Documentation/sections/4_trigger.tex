\chapter{Trigger Implementation}

\subsection*{CheckOrderInsertOrUpdate}
This trigger enforces the following logical constraints on orders:
\begin{itemize}[label=-]
    \item Feedback cannot be provided without a completion date.
    \item A team must be assigned before the completion date.
    \item The team cannot be changed after the order is completed.
    \item All employees in the same order must belong to the same team.
    \item Employees associated with a completed order cannot be updated.
    \item If an order has employees but no team, assign the team based on the employees' team.
    \item Ensures that feedback score, if feedback is provided, is not null.
\end{itemize}
\begin{lstlisting}
CREATE OR REPLACE TRIGGER CheckOrderInsertOrUpdate
BEFORE INSERT OR UPDATE ON OrderTB
FOR EACH ROW
DECLARE
    cnt NUMBER;
BEGIN
    
    IF :NEW.completionDate IS NULL THEN
        IF :NEW.feedback IS NOT NULL THEN
            RAISE_APPLICATION_ERROR(-20001, 'Feedback cannot be given without completion date');
        END IF;
    END IF;


    IF INSERTING THEN
        IF :NEW.team IS NULL AND :NEW.completionDate IS NOT NULL THEN
            RAISE_APPLICATION_ERROR(-20010, 'Team must be assigned before completion date');
        END IF;
    END IF;


    IF UPDATING THEN
        IF :OLD.completionDate IS NOT NULL AND 
            ((:OLD.team IS NULL AND :NEW.team IS NOT NULL) OR
            (:OLD.team IS NOT NULL AND :NEW.team IS NULL) OR
            (:OLD.team IS NOT NULL AND :NEW.team IS NOT NULL AND :NEW.team <> :OLD.team)) THEN
            RAISE_APPLICATION_ERROR(-20011, 'Team cannot be changed after order completion');
        END IF;
    END IF;


    IF :NEW.employees IS NOT NULL AND :NEW.employees.COUNT > 0 AND :NEW.team IS NOT NULL THEN
        SELECT COUNT(*) INTO cnt FROM TABLE(:NEW.employees) emp_ref
        JOIN EmployeeTB e ON (emp_ref.column_value = REF(e))
        WHERE e.team <> :NEW.team;

        IF cnt > 0 THEN
            RAISE_APPLICATION_ERROR(-20007, 'Employee of a different team in the same order detected');
        END IF;
    END IF;


    IF UPDATING THEN
        IF :OLD.completionDATE IS NOT NULL THEN
            IF :NEW.employees IS NOT NULL OR (:OLD.employees IS NOT NULL AND :NEW.employees IS NULL) THEN
                RAISE_APPLICATION_ERROR(-20008, 'Cannot update employees of a completed order');
            END IF;
        END IF;
    END IF;


    IF :NEW.team IS NULL AND :NEW.employees IS NOT NULL AND :NEW.employees.COUNT > 0 THEN
        SELECT e.team INTO :NEW.team
        FROM TABLE(:NEW.employees) emp_ref
        JOIN EmployeeTB e ON (emp_ref.column_value = REF(e))
        FETCH FIRST 1 ROW ONLY;
    END IF;


    IF :NEW.feedback IS NOT NULL AND :NEW.feedback.score IS NULL THEN
        RAISE_APPLICATION_ERROR(-20020, 'Feedback score cannot be null');
    END IF;
END;
\end{lstlisting}

\subsection*{CheckCustomerType}
Enforces that "individual" customers cannot have business data and vice versa:
\begin{lstlisting}
CREATE OR REPLACE TRIGGER CheckCustomerType
BEFORE INSERT OR UPDATE ON CustomerTB
FOR EACH ROW
BEGIN
    IF :NEW.type = 'individual' THEN
        IF :NEW.companyName IS NOT NULL THEN
            RAISE_APPLICATION_ERROR(-20002, 'Individual customers cannot have a company name');
        END IF;
        IF :NEW.address IS NOT NULL THEN
            RAISE_APPLICATION_ERROR(-20003, 'Individual customers cannot have an address');
        END IF;
    ELSIF :NEW.type = 'business' THEN
        IF :NEW.name IS NOT NULL THEN
            RAISE_APPLICATION_ERROR(-20004, 'Business customers cannot have a name');
        END IF;
        IF :NEW.surname IS NOT NULL THEN
            RAISE_APPLICATION_ERROR(-20005, 'Business customers cannot have a surname');
        END IF;
        IF :NEW.dob IS NOT NULL THEN
            RAISE_APPLICATION_ERROR(-20006, 'Business customers cannot have a date of birth');
        END IF;
    END IF;
END;
\end{lstlisting}

\subsection*{UpdateNumOrdersBeforeInsert}
Increments the numOrder attribute of a team before inserting a new order:
\begin{lstlisting}
CREATE OR REPLACE TRIGGER UpdateNumOrdersBeforeInsert
BEFORE INSERT ON OrderTB
FOR EACH ROW
BEGIN
    IF :NEW.team IS NOT NULL THEN
        UPDATE TeamTB t
            SET t.numOrder = t.numOrder + 1
            WHERE REF(t) = :NEW.team;
    END IF;
END;
\end{lstlisting} 

\subsection*{UpdateNumOrdersBeforeDelete}
Decrements the numOrder attribute of a team before deleting an order:
\begin{lstlisting}
CREATE OR REPLACE TRIGGER UpdateNumOrdersBeforeDelete
BEFORE DELETE ON OrderTB
FOR EACH ROW
BEGIN
    IF :OLD.team IS NOT NULL THEN
        UPDATE TeamTB t
            SET t.numOrder = t.numOrder - 1
            WHERE REF(t) = :OLD.team;
    END IF;
END; 
\end{lstlisting}

\subsection*{UpdateNumOrdersBeforeUpdate}
Updates the numOrder attribute of a team before updating an order:
\begin{lstlisting}
CREATE OR REPLACE TRIGGER UpdateNumOrdersBeforeUpdate
BEFORE UPDATE OF team ON OrderTB
FOR EACH ROW
BEGIN
    IF :OLD.team IS NOT NULL THEN
        UPDATE TeamTB t
            SET t.numOrder = t.numOrder - 1
            WHERE REF(t) = :OLD.team;
    END IF;

    IF :NEW.team IS NOT NULL THEN
        UPDATE TeamTB t
            SET t.numOrder = t.numOrder + 1
            WHERE REF(t) = :NEW.team;
    END IF;
END;
\end{lstlisting}

\subsection*{CheckTeamInsertInitialization}
Initializes numOrder and performanceScore in a correct way when inserting a new team (numOrder = 0, performanceScore = 1):
\begin{lstlisting}
CREATE OR REPLACE TRIGGER CheckTeamInsertInitialization
BEFORE INSERT ON TeamTB
FOR EACH ROW
BEGIN
    :NEW.numOrder := 0;
    :NEW.performanceScore := 1;
END;
\end{lstlisting}

\subsection*{CheckNumEmployeeInTeam}
Ensures a team cannot exceed 8 employees. This trigger makes use of a compound trigger to store the team reference and to avoid the mutating table problem:
\begin{lstlisting}
CREATE OR REPLACE TRIGGER CheckNumEmployeeInTeam
FOR INSERT OR UPDATE OF team ON EmployeeTB
COMPOUND TRIGGER
    cnt number;
    teamRef REF TeamTY;
BEFORE EACH ROW IS
BEGIN
    teamRef := :New.team;
END BEFORE EACH ROW;

AFTER STATEMENT IS
BEGIN
    SELECT COUNT(*) INTO cnt FROM EmployeeTB e WHERE e.team = teamRef;
    IF cnt > 8 THEN
        RAISE_APPLICATION_ERROR(-20001, 'Max number of employee reached');
    END IF;
    END AFTER STATEMENT;
END;
\end{lstlisting}

\subsection*{ComputePerformanceScore}
Recalculates a team's average feedback score whenever a new order is added, updated or inserted. It stores the team affected by the operation, and compute the new score of them:
\begin{lstlisting}
CREATE OR REPLACE TRIGGER ComputePerformanceScore
FOR INSERT OR UPDATE OR DELETE ON OrderTB
COMPOUND TRIGGER
    changedTeams TeamRefList := TeamRefList();

BEFORE EACH ROW IS
BEGIN
    IF DELETING OR UPDATING THEN
        IF :OLD.feedback IS NOT NULL AND :OLD.team IS NOT NULL THEN
            changedTeams.EXTEND;
            changedTeams(changedTeams.LAST) := :OLD.team;
        END IF;
    END IF;

    IF INSERTING OR UPDATING THEN
        IF :NEW.feedback IS NOT NULL AND :NEW.team IS NOT NULL THEN
            changedTeams.EXTEND;
            changedTeams(changedTeams.LAST) := :NEW.team;
        END IF;
    END IF;
END BEFORE EACH ROW;

AFTER STATEMENT IS
BEGIN
    UPDATE TeamTB t
        SET t.performanceScore = (
                SELECT ROUND(AVG(o.feedback.score), 2)
                    FROM OrderTB o
                WHERE o.team = REF(t)
                    AND o.feedback IS NOT NULL
            )
        WHERE REF(t) IN (
                SELECT COLUMN_VALUE
                    FROM TABLE(changedTeams)
            );
END AFTER STATEMENT;

END;
\end{lstlisting}

\subsection*{CheckScore}
Handle the dangling references not caught by \texttt{ComputePerformanceScore}:
\begin{lstlisting}
CREATE OR REPLACE TRIGGER CheckScore
BEFORE INSERT OR UPDATE ON TeamTB
FOR EACH ROW
BEGIN
    IF :NEW.numOrder = 0 AND :NEW.performanceScore != 1 THEN
        :NEW.performanceScore := 1;
    END IF;
END;
\end{lstlisting}

\subsection*{AddAccount}
Automatically creates a business account for every new customer:
\begin{lstlisting}
CREATE OR REPLACE TRIGGER AddAccount
FOR INSERT ON CustomerTB
COMPOUND TRIGGER
    customer VARCHAR2(11);
BEFORE EACH ROW IS
BEGIN
    customer := :NEW.VAT;
END BEFORE EACH ROW;

AFTER STATEMENT IS
    v_code VARCHAR2(10);
    v_exists NUMBER;
BEGIN
    insert into BusinessAccountTB values (
        sys_guid(),
        sysdate, 
        (SELECT REF(c) FROM CustomerTB c WHERE c.VAT = customer)
    );
END AFTER STATEMENT;
END;
\end{lstlisting}

\subsection*{DeleteTeamAfterOperationalCenter}
Deletes teams that lose their operational center reference upon an operational center's deletion:
\begin{lstlisting}
CREATE OR REPLACE TRIGGER DeleteTeamAfterOperationalCenter
AFTER DELETE ON OperationalCenterTB
BEGIN
    DELETE FROM TeamTB t
    WHERE DEREF(t.operationalCenter) IS NULL AND t.operationalCenter IS NOT NULL;
END;
\end{lstlisting}

\subsection*{UpdateEmployeeAfterTeam}
Sets an employee's team reference to NULL if the team is deleted, and unassigned orders if they're not completed:
\begin{lstlisting}
CREATE OR REPLACE TRIGGER UpdateEmployeeAfterTeam
AFTER DELETE ON TeamTB
BEGIN
    UPDATE EmployeeTB e
    SET e.team = NULL
    WHERE DEREF(e.team) IS NULL AND e.team IS NOT NULL;

    UPDATE OrderTB o
    SET o.team = NULL
    WHERE DEREF(o.team) IS NULL AND o.team IS NOT NULL AND o.completionDate IS NULL;
END;
\end{lstlisting}

\subsection*{DeleteAccountAfterCustomer}
Deletes business accounts that become orphaned when their customer is removed:
\begin{lstlisting}
CREATE OR REPLACE TRIGGER DeleteAccountAfterCustomer
AFTER DELETE ON CustomerTB
BEGIN
    DELETE FROM BusinessAccountTB ba
    WHERE DEREF(ba.customer) IS NULL AND ba.customer IS NOT NULL;
END;
\end{lstlisting}

\subsection*{DeleteOrdersAfterTeam}
Removes orders that have lost their team and business account references:
\begin{lstlisting}
CREATE OR REPLACE TRIGGER DeleteOrdersAfterTeam
AFTER DELETE ON TeamTB
BEGIN
    DELETE FROM OrderTB o
    WHERE DEREF(o.team) IS NULL AND o.team IS NOT NULL AND DEREF(o.businessAccount) IS NULL AND o.businessAccount IS NOT NULL;
END;
\end{lstlisting}

\subsection*{DeleteOrdersAfterAccount}
Deletes orders that have lost their related business account, under certain conditions:
\begin{itemize}[label=-]
    \item The business account reference is null and the completion date is null.
    \item The business account reference is null and the team reference is null.
\end{itemize}
In either case, there is no possibility of having useful information to compute the performance score, and there is no need to retain the order history because the order is not associated with any customer.
\begin{lstlisting}
CREATE OR REPLACE TRIGGER DeleteOrdersAfterAccount
AFTER DELETE ON BusinessAccountTB
BEGIN
    DELETE FROM OrderTB o
    WHERE DEREF(o.businessAccount) IS NULL AND o.businessAccount IS NOT NULL AND o.completionDate IS NULL;

    DELETE FROM OrderTB o
    WHERE DEREF(o.businessAccount) IS NULL AND o.businessAccount IS NOT NULL AND (o.team IS NOT NULL AND DEREF(o.team) IS NULL);
END;
\end{lstlisting}

\subsection*{PreventOrderDeletion}
Prevents order deletion unless it has lost all references or is uncompleted with no account:
\begin{lstlisting}
CREATE OR REPLACE TRIGGER PreventOrderDeletion
BEFORE DELETE ON OrderTB
FOR EACH ROW
DECLARE
    v_team TeamTY;
    v_account BusinessAccountTY;
BEGIN
    IF :OLD.team IS NOT NULL THEN
        SELECT DEREF(:OLD.team) INTO v_team FROM DUAL;
    END IF;
    IF :OLD.businessAccount IS NOT NULL THEN
        SELECT DEREF(:OLD.businessAccount) INTO v_account FROM DUAL;
    END IF;

    IF NOT (
        (:OLD.team IS NOT NULL AND v_team IS NULL AND 
            :OLD.businessAccount IS NOT NULL AND v_account IS NULL)
        OR
        (:OLD.businessAccount IS NOT NULL AND v_account IS NULL AND 
            :OLD.completionDate IS NULL)
    ) THEN
        RAISE_APPLICATION_ERROR(-20019, 'Order deletion not allowed in this case');
    END IF;
END;
\end{lstlisting}

\subsection*{EmptyEmployeeListAfterTeamUpdate}
Clears the employee list if a team reference changes:
\begin{lstlisting}
CREATE OR REPLACE TRIGGER EmptyEmployeeListAfterTeamUpdate
BEFORE UPDATE OF team ON OrderTB
FOR EACH ROW
BEGIN
    IF :OLD.team IS NOT NULL AND :NEW.team IS NOT NULL AND :OLD.team != :NEW.team THEN
        :NEW.employees := NULL;
    END IF;
END;
\end{lstlisting}
